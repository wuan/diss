\newif\ifpdf
\ifx\pdfoutput\undefined
    \pdffalse
\else
    \pdfoutput=1
    \pdftrue
\fi

\usepackage[english,german,ngerman]{babel}
\usepackage[latin1]{inputenc}
\usepackage[T1]{fontenc}
\usepackage[mtpluscal,mtbold]{mathtime}
%\usepackage{bm}
\usepackage{amsmath}
\usepackage{amssymb}
\usepackage{nicefrac}
\usepackage{units}
\usepackage{graphicx}
\usepackage[tight,hang,raggedright]{subfigure}
\usepackage{float}
\usepackage{caption2}
\usepackage{calc}
  \renewcommand{\captionlabelfont}{\bfseries}
  \renewcommand{\captionfont}{\small\slshape}

\ifpdf 
	\pdfcompresslevel9
	\usepackage{thumbpdf}
	\usepackage[pdftex,
        colorlinks=true,
        urlcolor=rltred,       % \href{...}{...} external (URL)
        filecolor=rltgreen,     % \href{...} local file
        linkcolor=rltblue,       % \ref{...} and \pageref{...}
        citecolor=rltgreen,
        pdftitle={Dissertation: Surface Electrons on thin helium films},
        pdfauthor={Andreas Würl},
        pdfsubject={2DES on thin liquid helium films, exploration of the phase diagram},
        pdfkeywords={2DES, helium, films, electrons, Andreas Wuerl, Andreas Würl, Andi Würl},
        pagebackref,
        pdfpagemode=None,
        bookmarksopen=true]{hyperref}
	\usepackage{color}
	\definecolor{rltred}{rgb}{0.4,0,0}
	\definecolor{rltgreen}{rgb}{0,0.5,0}
	\definecolor{rltblue}{rgb}{0,0,0.5}
	\newcommand{\datalink}[2]{\href{http://wuerl.net/diss/microwave/#1}{#2}}
	\newcommand{\plotlink}[2]{\href{http://wuerl.net/diss.html/#1.gp}{#2}}
	\newcommand{\link}[1]{\href{http://#1}{#1}}
\else
	\newcommand{\datalink}[2]{#2}
	\newcommand{\plotlink}[2]{#2}
	\newcommand{\link}[1]{#1}
\fi
  
\bibliographystyle{leidiss}

%
% Neudefinitonen von Kommandos/Umgebungen
%

\renewcommand{\vec}{\bold}

\newcommand{\siehe}{\ensuremath{\nearrow}}
\newcommand{\hae}{\marginpar{? hae ?}}
\newcommand{\TODO}{\marginpar{TODO}}

\newcommand{\e}{\ensuremath{e^\text{-}}}
\newcommand{\SiO}{\ensuremath{\text{SiO}_2}}
\newcommand{\DC}{\ensuremath{\Delta C_3}}
\newcommand{\Ecm}{\ensuremath{\text{cm}^{-2}}}
\newcommand{\Em}{\ensuremath{\text{\upshape m}^{-2}}}
\newcommand{\kB}{\ensuremath{k_\text{B}}}
\newcommand{\HR}{Hohlraumresonator}
\newcommand{\RF}{Resonanzfrequenz}
\newcommand{\DK}{Dielektrizitätskonstante}
\newcommand{\ea}{{\itshape et al.}}
\newcommand{\name}[1]{{\scshape #1}}
\newcommand{\grmu}{\textgr{m}}

%
% Hier die Definitionen für einen Befehl, \noref, der zwar die Bezeichnung 
\makeatletter
\def\@nosetref#1#2#3{%
  \ifx#1\relax
   \protect\G@refundefinedtrue
   \nfss@text{\reset@font\bfseries ??}%
   \@latex@warning{Reference `#3' on page \thepage \space
             undefined}%
  \else
   \expandafter#2#1\null
  \fi}
\def\noref#1{\expandafter\@nosetref\csname r@#1\endcsname\@firstoffive{#1}}
\makeatother
%
% Hier die Definitionen um einen nicht-kursiven griechischen Buchstaben im Text zu verwenden.
% Definiert den Befehl \textgr
\makeatletter
\DeclareFontEncoding{LmtG}{}{\noaccents@}
\DeclareFontFamily{LmtG}{mtg}{}
\DeclareFontShape{LmtG}{mtg}{\mddefault}{\updefault}{<->mtgu}{}
\DeclareFontShape{LmtG}{mtg}{\bfdefault}{\updefault}{<->mtgub}{}
\DeclareFontSubstitution{LmtG}{mtg}{\mddefault}{\updefault}
\def\greekshape{\fontencoding{LmtG}\selectfont}%
\DeclareTextFontCommand{\textgr}{\greekshape}
\makeatother

% H"aufig benötigte Worte/Floskeln
% Bildbreiten:
\newlength{\ssmallwidth}
\setlength{\ssmallwidth}{0.4\textwidth}
\newlength{\smallwidth}
\setlength{\smallwidth}{0.495\textwidth}
\newlength{\smidwidth}
\setlength{\smidwidth}{0.6\textwidth}
\newlength{\midwidth}
\setlength{\midwidth}{0.7\textwidth}
\newlength{\bigwidth}
\setlength{\bigwidth}{0.9\textwidth}

%%%%%%%%%%%%%%%%%%%%%%%%%%%%%%%%%%%%%%%%%%%%%%%

% quer: Setzt Strich "uber #1
\newcommand{\quer}[1]{\overline{#1}}
% ul: Unterstreicht #1
\newcommand{\ul}[1]{\ensuremath{\underline{\mathrm{#1}}}}

% Text in Formeln mit Raum vor (\beforetext) und nach (\aftertext) dem Text
\newcommand{\beforetext}{\quad}
\newcommand{\aftertext}{\quad}
\newcommand{\ttext}[1]{\beforetext\text{#1}} % Abstand nur vorher
\newcommand{\textt}[1]{\text{#1}\aftertext}  % Abstabd nur nachher
\newcommand{\ttextt}[1]{\beforetext\text{#1}\aftertext} % Abstand vor- und nachher

% "Betrag": schlie"st #1 in angepaßte Striche | ein
\newcommand{\abs}[1]{\left\lvert \smash[t]{#1} \right\rvert}
\newcommand{\brak}[1]{\left<#1\right>}

% \log-like
\DeclareMathOperator{\sgn}{sgn}

%%%%%%%%%%%%%%%%%%%%%%%%%%%%%%%%%%%%%%%%%%%%%%%%%%%%%%%%%%%%%%%%%%%%%%%%%%%%%%%%%%%%%%%%%%%%%%%%%%
%\setlength{\parindent}{0em} \setlength{\parskip}{1ex}

\newenvironment{explanation}
    {\footnotesize$$\begin{tabular}{r@{ : }l}}
    {\end{tabular}$$}

%\setcounter{tocdepth}{3}
\pagestyle{headings}
