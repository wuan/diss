\section*{Danksagung}
\markboth{Danksagung}{Danksagung}
\addcontentsline{toc}{chapter}{Danksagung}
Dank gilt natürlich allen, die -- wie auch immer -- zur Fertigstellung dieser Arbeit beigetragen haben. Speziell hervorheben möchte ich dabei:
\begin{itemize}
	\item Meinen Doktorvater, Herrn Prof.~Dr.~P.~Leiderer, für die Möglichkeit diese Arbeit in seiner vielseitigen Arbeitsgruppe anfertigen zu können, für die dabei überlassenen Freiräume und Möglichkeiten, die vielen anregenden Diskussionen über das Experiment und sein umfassendes physikalisches Wissen in Verbindung mit guter Intuition.
	\item Dr.~Jürgen Klier für die direkte Betreuung der Arbeit in der Tieftemperaturgruppe des Lehrstuhls, die fruchtbaren Diskussionen und hilfreichen Tipps sowie die Motivierung -- auch wenn sich die Messdaten mal nicht in die gewünschte Richtung bewegten.
	\item Valeri~Shikin für die produktive Zusammenarbeit und die Gespräche -- auch über die Grenzen von Theorie und Experiment hinweg -- die einige Teilaspekte dieser Arbeit erst hervorgebracht haben.
	\item Kimitoshi~Kono für die Starthilfe auf dem Gebiet der Elektronensysteme auf Helium, die er mir durch den Aufenthalt in seiner Arbeitsgruppe ermöglichte.
	\item Allen Mitgliedern des Lehrstuhls Leiderer für die angenehme Atmosphäre und Hilfsbereitschaft sowie für den wichtigen Austausch von Wissen, benötigten Geräten und Messdienstleistungen über die Grenzen der Arbeitsgruppen hinweg.
	\item Den "`Bürokollegen"' Hajo, Volker und Konrad für die vielen gemeinsamen Erlebnisse, die gegenseitige Unterstützung in Computerfragen und die allgemeine Zerstreuung jenseits der Physik.
	\item Hajo, Kalli, Claudi, Volker, Laurent und Masoud für die Ablenkung in Form von schönen gemeinsamen Bergtouren, Skiausflügen und Radltouren sowie der einen oder anderen Abendstunde am Kletterturm der Uni.
	\item Konrad für die immerwährende Gastfreundschaft und die Hilfe "`vor Ort"' vor allem in der Endphase der Arbeit.
	\item Allen fleißigen Korrekturlesern, insbesondere Gerhard, für den unerbittlichen Kampf gegen den Fehlerteufel.
	\item Die deutsche Forschungsgemeinschaft DFG für die Finanzierung dieser Arbeit durch die Forschergruppe "`Quantengase"'.
	\item Die Firma Wacker Siltronic Burghausen für die Bereitstellung von in den Experimenten verwendeten Silizium-Wafern.
	\item In besonderer Weise meine Eltern für die vielfältige Unterstützung während der Arbeit und vor allem natürlich, dass sie mir mein Studium ermöglicht haben. 
\end{itemize}
