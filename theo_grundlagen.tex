\documentclass[pointlessnumbers,headsepline,twoside,11pt,DIV14,BCOR15mm,halfparskip,a4paper,appendixprefix]{scrreprt}

\usepackage[english,german,ngerman]{babel}
\usepackage[utf8]{inputenc}
\usepackage[T1]{fontenc}
%\usepackage{bm}
\usepackage{amsmath}
\usepackage{amssymb}

\let\Bbbk\relax
\usepackage{times}
\usepackage[subscriptcorrection]{mtpro2}

\usepackage{nicefrac}
\usepackage{units}
\usepackage{graphicx}
\usepackage[tight,hang,raggedright]{subfigure}
\usepackage{float}
\usepackage[normal]{caption}
\usepackage{calc}
  \renewcommand{\captionlabelfont}{\bfseries}
  \renewcommand{\captionfont}{\small\slshape}
\usepackage{gensymb}
\usepackage{bibgerm}

\pdfcompresslevel9
\usepackage{thumbpdf}
\usepackage{microtype}

\usepackage[pdftex,
	colorlinks=true,
	urlcolor=rltred,       % \href{...}{...} external (URL)
        filecolor=rltgreen,     % \href{...} local file
        linkcolor=rltblue,       % \ref{...} and \pageref{...}
        citecolor=rltgreen,
        pdftitle={Hochfrequenzuntersuchung an zweidimensionalen Elektronensystemen auf dünnen Heliumfilmen},
        pdfauthor={Andreas Würl},
        pdfsubject={},
        pdfkeywords={surface electrons, helium films, phase diagram, quantum melting},
        pagebackref,
        pdfpagemode=None,
    bookmarksopen=true]{hyperref}
\usepackage{color}
\definecolor{rltred}{rgb}{0.4,0,0}
\definecolor{rltgreen}{rgb}{0,0.5,0}
\definecolor{rltblue}{rgb}{0,0,0.5}
\newcommand{\datalink}[2]{\href{http://www.wuerl.net/diss/microwave/#1}{#2}}
\newcommand{\plotlink}[2]{\href{http://www.wuerl.net/diss/gp/#1.gp}{#2}}
\newcommand{\link}[1]{\href{http://#1}{#1}}
  
\bibliographystyle{leidiss}

%
% Neudefinitonen von Kommandos/Umgebungen
%

\renewcommand{\vec}{\bold}

\newcommand{\siehe}{\ensuremath{\nearrow}}
\newcommand{\hae}{\marginpar{? hae ?}}
\newcommand{\TODO}{\marginpar{TODO}}

\newcommand{\e}{\ensuremath{e^\text{-}}}
\newcommand{\SiO}{\ensuremath{\text{SiO}_2}}
\newcommand{\DC}{\ensuremath{\Delta C_3}}
\newcommand{\Ecm}{\ensuremath{\text{cm}^{-2}}}
\newcommand{\Em}{\ensuremath{\text{\upshape m}^{-2}}}
\newcommand{\kB}{\ensuremath{k_\text{B}}}
\newcommand{\HR}{Hohlraumresonator}
\newcommand{\RF}{Resonanzfrequenz}
\newcommand{\DK}{Dielektrizitätskonstante}
\newcommand{\ea}{{\itshape et al.}}
\newcommand{\name}[1]{{\scshape #1}}
%\newcommand{\grmu}{\textgr{m}}
\newcommand{\grmu}{\ensuremath{\mu}}

%
% Hier die Definitionen für einen Befehl, \noref, der zwar die Bezeichnung 
\makeatletter
\def\@nosetref#1#2#3{%
  \ifx#1\relax
   \protect\G@refundefinedtrue
   \nfss@text{\reset@font\bfseries ??}%
   \@latex@warning{Reference `#3' on page \thepage \space
             undefined}%
  \else
   \expandafter#2#1\null
  \fi}
\def\noref#1{\expandafter\@nosetref\csname r@#1\endcsname\@firstoffive{#1}}
\makeatother
%
% Hier die Definitionen um einen nicht-kursiven griechischen Buchstaben im Text zu verwenden.
% Definiert den Befehl \textgr
\makeatletter
\DeclareFontEncoding{LmtG}{}{\noaccents@}
\DeclareFontFamily{LmtG}{mtg}{}
\DeclareFontShape{LmtG}{mtg}{\mddefault}{\updefault}{<->mtgu}{}
\DeclareFontShape{LmtG}{mtg}{\bfdefault}{\updefault}{<->mtgub}{}
\DeclareFontSubstitution{LmtG}{mtg}{\mddefault}{\updefault}
\def\greekshape{\fontencoding{LmtG}\selectfont}%
\DeclareTextFontCommand{\textgr}{\greekshape}
\makeatother

% H"aufig benötigte Worte/Floskeln
% Bildbreiten:
\newlength{\ssmallwidth}
\setlength{\ssmallwidth}{0.4\textwidth}
\newlength{\smallwidth}
\setlength{\smallwidth}{0.495\textwidth}
\newlength{\smidwidth}
\setlength{\smidwidth}{0.6\textwidth}
\newlength{\midwidth}
\setlength{\midwidth}{0.7\textwidth}
\newlength{\bigwidth}
\setlength{\bigwidth}{0.9\textwidth}

%%%%%%%%%%%%%%%%%%%%%%%%%%%%%%%%%%%%%%%%%%%%%%%

% quer: Setzt Strich "uber #1
\newcommand{\quer}[1]{\overline{#1}}
% ul: Unterstreicht #1
\newcommand{\ul}[1]{\ensuremath{\underline{\mathrm{#1}}}}

% Text in Formeln mit Raum vor (\beforetext) und nach (\aftertext) dem Text
\newcommand{\beforetext}{\quad}
\newcommand{\aftertext}{\quad}
\newcommand{\ttext}[1]{\beforetext\text{#1}} % Abstand nur vorher
\newcommand{\textt}[1]{\text{#1}\aftertext}  % Abstabd nur nachher
\newcommand{\ttextt}[1]{\beforetext\text{#1}\aftertext} % Abstand vor- und nachher

% "Betrag": schlie"st #1 in angepaßte Striche | ein
\newcommand{\abs}[1]{\left\lvert \smash[t]{#1} \right\rvert}
\newcommand{\brak}[1]{\left<#1\right>}

% \log-like
\DeclareMathOperator{\sgn}{sgn}

%%%%%%%%%%%%%%%%%%%%%%%%%%%%%%%%%%%%%%%%%%%%%%%%%%%%%%%%%%%%%%%%%%%%%%%%%%%%%%%%%%%%%%%%%%%%%%%%%%
%\setlength{\parindent}{0em} \setlength{\parskip}{1ex}

\newenvironment{explanation}
    {\footnotesize$$\begin{tabular}{r@{ : }l}}
    {\end{tabular}$$}

%\setcounter{tocdepth}{3}
\pagestyle{headings}


\begin{document}
\chapter{Theoretische Grundlagen}
	\label{chap:theo}
\section{Leitfähigkeitsmodell nach Drude}
\subsection{Grundlegende Annahmen}
Das Modell von \name{Drude} ist die Anwendung der kinetischen Gastheorie auf die Elektronen in einem Leiter, das aus der Vorstellung von einem Elektronen-"`Gas"' herrührt. Mit einigen Vereinfachungen kann man viele bekannte Effekte damit beschreiben:
\begin{itemize}
    \item Zwischen Kollisionen ist die Wechselwirkung von Elektronen mit anderen Elektronen und den Atomrümpfen vernachlässigbar. Jedes Elektron bewegt sich geradlinig. Externe Felder wirken direkt, interne werden vernachlässigt.

    \item Kollisionen sind plötzliche Ereignisse, die instantan die Geschwindigkeit und die Bewegungsrichtung eines Elektrons ändern. Hauptsächlich spielt Elektron-Ionen-Streuung eine Rolle.

    \item Die Wahrscheinlichkeit, dass ein Stoß stattfindet, skaliert mit $\frac1\tau$ ($\tau$ ist die Kollisionszeit). Ein Elektron wird sich im Mittel für eine Zeit $\tau$ bewegen, bevor es gestreut wird. $\tau$ ist unabhängig von der Position und der Geschwindigkeit des Elektrons.

    \item Elektronen erreichen das thermische Gleichgewicht mit der Umgebung nur durch Energieaustausch bei Stößen mit Ionenrümpfen.
\end{itemize}

\subsection{Herleitung für 3D}
Das Ohmsche Gesetz
    \begin{equation}
        V = I R
    \end{equation}
kann man für geladene Teilchen folgendermaßen schreiben
    \begin{equation}
        \vec E = \rho \vec j\quad.
    \end{equation}
Hierbei ist $\vec E$ das elektrische Feld, $\rho$ die Resistivität und
$\vec j=\frac{Q}{tA}$ die Stromdichte, für die man bei einem gleichförmigen Strom auch $j = \frac{I}{A}$ schreiben kann. Für eine Potentialdifferenz $V$ erhält man dann
    \begin{equation}
        V = E l = \rho \frac{I}{A} l\quad\Rightarrow\quad R = \frac{\rho l}{A}\quad.
    \end{equation}
Für Elektronen, die sich mit der Geschwindigkeit $\vec v$ bewegen, ist $\vec j\parallel\vec v$. In der Zeit $dt$ bewegen sich die Elektronen um $\vec v dt$ in Richtung von $\vec v$. Die Elektronen durchfließen eine Fläche $A$ senkrecht zu $\vec v$, wobei jedes Elektron die Ladung $-e$ trägt. Die Ladung die $A$ in der Zeit $dt$ durchfließt ist dann $-n e v A\,dt$, daher ergibt sich die Stromdichte zu
    \begin{equation}
        \label{eqn:drude_stromdichte}
        \vec j = - n e \vec v\quad.
    \end{equation}
Ohne äußeres elektrisches Feld ist $\vec v$ im Mittel gleich Null. Bei Anwesenheit eines elektrischen Feldes werden die Elektronen beschleunigt. Stellen wir uns ein Elektron zur Zeit $t=0$ vor; $t$ ist dann die Zeit, die seit der letzten Kollision vergangen ist. Ein äußeres Feld ändert die Geschwindigkeit um $-e\vec E\frac{t}{m}$, was sich aus
    \begin{equation}
        F=ma\quad\Rightarrow\quad a=\frac{F}{m}\quad\Rightarrow\quad v=at=\frac{Ft}{m}\quad.
    \end{equation}
ergibt. Daher ist die durchschnittliche Geschwindigkeit durch
    \begin{equation}
        \vec v_\text{avg}=-\frac{e\vec E\tau}{m}
    \end{equation}
gegeben. Eingesetzt in Gleichung~\eqref{eqn:drude_stromdichte} ergibt sich das Drude-Gesetz für Leitung von Gleichstrom in einem metallischen Leiter
	\begin{equation}
	\vec j = \sigma\vec E\ttextt{mit:}\sigma=\frac{ne^2\tau}{m}\quad.
	\end{equation}
Alle Größen bis auf die Stoßzeit $\tau$ sind bekannt; diese kann man nun aus den beobachteten Resistivitäten bestimmen
	\begin{equation}
	\tau=\frac{m}{\rho ne^2}=\frac{\sigma m}{ne^2}\quad.
	\end{equation}
Bei Metallen und Raumtemperatur liegt $\tau$ typischerweise im Bereich von $10^{-14}$ bis $\unit[10^{-15}]{s}$.

Für die Wechselstromleitfähigkeit setzt man ein zeitlich sinusförmiges elektrisches Feld in die Bewegungsgleichung für den Impuls $\vec p$ der Elektronen ein:
	\begin{equation}
		\frac{d\vec p}{dt}=-\frac{\vec p}{\tau}-e\vec E\quad.
	\end{equation}
Aus $\vec j=-n\,e\,\vec p/m_e$ und der frequenzabhängigen Stromdichte $\vec j(\omega)=\sigma(\omega)\vec E(\omega)$ folgt dann:
	\begin{equation}
		\sigma(\omega)=\frac{\sigma_0}{1-i\omega\tau}=
			\frac{n\,e^2\tau}{m_e}\left[\frac1{1+\omega^2\tau^2}+
				i\frac{\omega\,\tau}{1+\omega^2\tau^2}\right]
		\ttextt{mit}\sigma_0=\frac{n\,e^2\tau}{m_e}
	\end{equation}
\end{document}
