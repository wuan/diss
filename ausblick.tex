\chapter{Ausblick}

Mit den oxidierten Silizium"=Substraten mit \unit[200]{nm} \SiO"=Schicht konnten in den durchgeführten Experimenten mit Abstand die höchsten Elektronendichten und auch die größte, nur vom 2DES verursachte, Variation der Transmission gemessen werden. Leider war es bei einer Reihe von Versuchen mit diesen Substraten nicht möglich, nach dem Abkühlen ein Resonanzsignal des mit dem Siliziumplättchen bestückten Resonators zu messen. Dieses Verhalten kann durch im Substrat eingeschlossene Verunreinigungen erklärt werden, die im verwendeten Frequenzbereich eine hohe Absorption besitzen. Eine weitere Möglichkeit der Erklärung des beobachteten Verhaltens ist eine elektrische Aufladung der Substratoberfläche bereits vor dem Einbau, welche zu lokal erhöhter Leitfähigkeit und somit wieder zu Absorption der Mikrowellen führt.

Durch eine unterschiedliche Vorbehandlung der Substrate, wie z.~B. durch gezieltes Aufbringen von Oberflächenladungen mit dem Plasmacleaner oder durch Lagerung der Substrate in Essigsäuredampf wurde versucht die Quelle der nicht erwünschten Mikrowellenabsorption aufzudecken.  Es konnte jedoch nicht abschließend geklärt werden, welche Ursachen dieses, bei vielen Messungen beobachtete, Verhalten bewirken. 

Nach Lösung derartiger Probleme sind systematische Messungen mit \SiO"=Substraten sehr wichtig um, wie schon in Abschnitt~\ref{sec:film_messpar} verdeutlicht, die genauen Parameter für einen optimalen Beladeprozess zu bestimmen. 

Gerade bei den in direkter Folge ausgeführten Beladezyklen ist eine Kontrolle der Reproduzierbarkeit und die deutliche Erkennung des Beginns der Beladung der Substratoberfläche mit Elektronen sehr leicht möglich. Die Kenntnis dieser Begleiterscheinungen und deren zweifelsfreie Erkennung ist deshalb eine wichtige Voraussetzung für die Relevanz der erhaltenen Messdaten.

Wenn all diese Voraussetzungen erfüllt sind ist das System 2DES auf Helium auf mit \SiO{} bedecktem Silizium ein guter Kandidat hohe Elektronendichten und auch Beweglichkeiten der Elektronen im 2DES in Verbindung mit einem guten Signal"=Rausch"=Verhältnis der Messdaten zu erreichen. Erst wenn Dichten im Bereich von \unit[$10^{15}$]{\Em} routinemäßig erreicht werden können sollte es zweifelsfrei möglich sein das Verhalten des Systems zu erklären.

Silizium"=Substrate mit der natürlichen Oxidschicht oder sogar Substrate mit entfernter Oxidschicht sind für das Erreichen der hohen Elektronendichten nur bedingt geeignet. Wie man in Abbildung~\ref{fig:film_selbstkonsistent} sehen kann ist die Filmdickenerniedrigung bei der Beladung zu hohen Dichten hin so stark, dass das Einsetzen von Tunnelprozessen der Elektronen bei einer Filmdicke in der Größenordnung von einigen Nanometern die erreichbare Elektronendichte begrenzt.

\subsection*{Verwendung neuer Substratmaterialien}
Die Variante, Kohlenstoff als Elektrode auf beliebige, isolierende Substrate aufzubringen eröffnet völlig neue Möglichkeiten für weitere Experimente. Man kann z.~B. durch eine Aufdampfmaske eine bestimmte Form und somit ein räumlich eingeschränktes Elektronensystem realisieren. Weiterhin ist es hiermit möglich eine in Abschnitt~\ref{ssec:elektronenverlust} Guard"=Elektrode direkt mit auf das Substrat aufzubringen.
Falls nach den hier durchgeführten ersten Versuchen Kohlenstofffilme als Substratelektrode weiterhin erfolgreich sind, bedeutet dies, dass eine große Klasse von Materialien mit glatten, isolierenden Oberflächen, wie z.~B. Saphir als Substrat für die Experimente zur Verfügung stehen. Auch lässt sich die Funktion des experimentellen Aufbaus hier schon bei Raumtemperatur besser überwachen, da die Kohlenstoffilme in diesem Temperaturbereich die Resonanz des Hohlraumresonators nicht so stark dämpfen, wie es im Gegensatz dazu bei der Verwendung von Substraten aus Silizium der Fall ist.
Das Erreichen von sehr hohen Elektronendichten im Bereich von mehr als \unit[$10^{15}$]{\Em} wird hier auf Grund der bei Filmdickenerniedrigung einsetzenden Tunnelprozessen von Elektronen in das leitfähige Substrat problematisch sein.
