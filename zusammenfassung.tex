\markboth{Zusammenfassung}{Zusammenfassung}
\chapter*{Zusammenfassung}
\addcontentsline{toc}{chapter}{Zusammenfassung}

Im Rahmen dieser Arbeit wurde als Alternative zur bisher verwendete Messmethode durch einen LockIn"=gesteuerten Regelkreis (Beschreibung im Abschnitt~\ref{ssec:old_method}) eine Messmethode unter Verwendung eines Netzwerkanalysators mit direkter Detektion der Resonanzlinienparameter entwickelt (siehe Abschnitt~\ref{ssec:new_method}). Diese neue Messmethode hat den großen Vorteil, dass man zu jedem Zeitpunkt der Messung alle Parameter der Resonanzlinie kennt und daher unerwartete Veränderungen des Messsignals erkannt werden können. D.~h.~starke Änderungen der Resonatorgüte oder eine Verkippung der Resonanzkurve infolge von Leitungsreflexionen können zweifelsfrei erkannt werden.

Der Netzwerkanalysator bestimmt die Transmission des Systems für eine diskrete Anzahl von Frequenzpunkten. Um mit dieser Methode eine mit dem bisherigen Aufbau vergleichbare Genauigkeit zu erreichen ist es nicht ausreichend die maximale Transmission aus diesen Punkten zu bestimmen. Man muss immer einen Kompromiss zwischen der Dauer eines Messzyklus und der dabei erreichten Auflösung, gegeben durch die Anzahl von Messpunkten, schließen. Weiterhin hat sich gezeigt, dass eine Kurvenanpassung der gemessenen Transmissionswerte an die Resonanzfunktion des Hohlraumresonators erforderlich ist, die alle in einer gemessenen Transmissionswerte über einen bestimmten Frequenzbereich mit einbezieht und so die hohe Genauigkeit der erhaltenen Parameter der Resonanzlinie sicherstellt.

Diese Anforderung bedeutet jedoch, dass die Regelung auf das Resonanzmaximum langsamer als mit dem bisherigen Aufbau erfolgt, da für eine Messung der Lage der Resonanzlinie die Zeit zur Bestimmung der einzelne Messpunkte aufgebracht werden muss. Zur Messung von schnellen Prozessen mit einer Zeitauflösung ab \unit[2]{s} ist es deshalb besser den vorhandenen LockIn"=gesteuerten Regelkreis einzusetzen. Weiterhin ist es mit dem neuen Aufbau schwieriger eine manuell kontrollierte Messung durchzuführen, da die Linienanpassung immer von einem Rechner durchgeführt werden muss.

Es wurde nachgewiesen, dass die Genauigkeit und das erreichte Signal"=Rausch"=Verhältnis für beide Messmethoden im Rahmen der erreichbaren Messbedingungen vergleichbar ist und im Prinzip nur durch das von der Umgebung im Messsystem erzeugten Rauschen abhängt.

Im zweiten Teil der Arbeit wurde das von \name{V. Shikin} entwickelte Zweikomponentenmodell für Elektronen auf dünnen Heliumfilmen auf experimentelle Ergebnisse der Messung der Zyklotronresonanz an 2DES angewendet. Dieses Modell macht eine indirekte Bestimmung der Oberflächenparameter des Substrats im Experiment möglich. Dies geschieht durch die Einführung eines weiteren physikalisch motivierten Fitparameters, der das Verhältnis von freien und lokalisierten Elektronen widerspiegelt. Die Magnetfeldabhängigkeit von bereits vorhandene Messdaten von \name{G. Mistura} und auch die eigens aufgenommener Messdaten konnten mit diesem Modell sehr gut beschrieben werden. Die aus dem Modell erhaltenen Daten können in Parameter eines Modellsubstrats umgerechnet werden, die wiederum als wichtige Grundlage bei der entscheidenden Optimierung der Qualität der Substratoberfläche dienen können.
 
Die aus den durchgeführten Messungen erhaltenen Erkenntnisse sind im Folgenden skizziert:

Eine große Bedeutung in dieser Arbeit hat die Erzeugung sehr hoher Elektronendichten im 2DES. Als Folge der Existenz der elektrohydrodynamischen Instabilität können solche Messungen nicht auf Bulk"=Helium durchgeführt werden. Die Verwendeung der für hohe Elektronendichten erforderlichen dünnen Heliumfilme bedeutet dann allerdings einen starken Einfluss der Beschaffenheit der Substratoberfläche, die den Film trägt, auf die erhaltenen Messergebnisse.
Dies hat folgende Konsequenzen:
\begin{itemize}
	\item Die Rauigkeit der Substratoberfläche verringert hier die Beweglichkeit der Elektronen. Dadurch, dass der Helium"=Film auf dem Substrat sehr dünn ist und das Substrat eine im Vergleich zum Helium deutlich größere Dielektrizitätskonstante aufweist führt dies zu einer starken ortsabhängigen Schwankung der Spiegelladung der Elektronen und beeinflußt sehr stark das Verhalten der Elektronen im 2DES.
Die Beweglichkeit der Elektronen ist hier -- im Vergleich zum Fall von Elektronen auf Bulk"=Helium -- um Größenordnungen geringer. Um dennoch messbare Signale zur Bestimmung der Beweglichkeit der Elektronen zu erhalten empfiehlt sich hier die Verwendung hoher Frequenzen im Mikrowellenbereich.
	\item Das Tunneln von Elektronen verursacht einen mit abnehmender Filmdicke zunehmende Verlust von Elektronen, der bei hohen Elektronendichten und eine entscheidende Rolle spielt. Hier ist die Verwendung von Substraten mit isolierender Deckschicht ratsam.
\end{itemize}

