\chapter*{Einleitung}
\addcontentsline{toc}{chapter}{Einleitung}
\markboth{Einleitung}{Einleitung}

Als Sir Joseph John Thomson im Jahre 1897 das Elektron als das erste bekannte Elementarteilchen entdeckte, war dies der Beginn der mittlerweile sehr weit reichenden Elementarteilchenphysik und auch die Grundlage für alle nachfolgenden Entwicklungen in der Elektronik. Viele daraus enstandene Produkte erachten wir mittlerweile in vielen Teilen des täglichen Lebens als selbstverständlich. Als wichtige Meilensteine seien hier die Erfindung der Elektronenröhre für Verstärker und Anzeigeinstrumente, gefolgt von der Entwicklung des ersten Transistors im Jahre 1947 als Geburtsstunde der Halbleiterphysik angeführt. Die schnelle Weiterentwicklung in der Halbleitertechnik führte dann auch zu Halbleiter-Heterostrukturen im so genannten Feldeffekttransistor (FET), der die Stromleitung über die Einschränkung eines zweidimensionalen Ladungskanals von Elektronen durch ein elektrisches Feld steuern kann. 

Wenn aktuell der Begriff eines zweidimensionalen Elektronensystems (2DES) verwendet wird, ist damit in fast allen Fällen die Ausbildung einer zweidimensionalen Elektronenschicht in diesen speziellen Halbleiter"=Schichtstrukturen gemeint. Im Gegensatz dazu weithin unbekannt ist die Möglichkeit, ein freies 2DES oberhalb der Oberfläche von flüssigem Helium zu erzeugen. Dieses "`andere"' 2DES wurde zu Beginn der siebziger Jahre nach einer Idee von \name{W.~Shockley}~\cite{Sho39} unabhängig von \name{M.~W.~Cole} \cite{Col69} und \name{V.~Shikin} \cite{Shi70} vorhergesagt und bald darauf von \name{R.~Williams}, \name{R.~S.~Crandall} und \name{A.~H.~Willis} \cite{Wil71} experimentell bestätigt.

Die Elektronen eines 2DES im Halbleiter können in Grundzügen durch das Modell der Fermi"=Flüssigkeit beschrieben werden, da sich hier die kinetische Energie der Elektronen in derselben Größenordnung wie ihre Coloumb"=Wechselwirkung bewegt und die Elektronendichte sehr hoch ist. Die Fermienergie ist also die für das Verhalten entscheidende Energie. Bei den Elektronen auf flüssigem Helium befindet man sich üblicherweise jedoch in einem völlig anderen Bereich im Phasendiagramm des 2DES. Bei den in diesem System unter experimentellen Bedingungen herrschenden Elektronendichten und Temperaturen ist das Verhältnis der thermischen Energie der Elektronen zu Ihrer kinetischen Energie in der Größenordnung von 1. Im Gegensatz zum 2DES im Halbleiter ist die Elektronendichte hier in einem weiten Bereich einstellbar: Ausgehend von einem Zustand ohne 2DES bei verschwindenden Dichtewerten bis zu einer oberen Grenze bei höheren Elektronendichten, die abhängig von der Erzeugung des Systems auf Bulk"=Helium bei $\unit[2\times10^{13}]{\Em}$ oder auf dünnen Heliumfilmen in einem Bereich von mehr als $\unit[5\times10^{14}]{\Em}$ liegt. Im Halbleiter ist die Elektronendichte in gewissen Bereichen an das Material der Atomrümpfe des Halbleitermaterials gekoppelt und Verbiegungen der Bandstruktur als Einfluss auf die Leitungselektronen sind nur in gewissen Grenzen durchführbar.

Experimente am System Elektronen auf flüssigem Helium werden im Prinzip im Kontext der folgenden zwei Szenarien durchgeführt:
\begin{itemize}
	\item Ein 2DES aus Elektronen auf flüssigem Helium kann als ein {\itshape Modellsystem für ein sehr reines oder verdünntes 2DES} dienen. Da die Elektronendichte im Gegensatz zum Halbleiter beliebig verringert werden kann, ist es möglich, Messungen unter Bedingungen durchzuführen, wie sie dort nur schwer zu erreichen sind. Weiterhin kann man auf Bulk"=Helium bei Temperaturen unter \unit[1]{K} ein sehr reines System mit hochbeweglichen Elektronen erhalten.
	\item In der Literatur finden sich einige Beispiele für die {\itshape Verwendung des 2DES als hochempfindliche Sonde} für die Beschaffenheit der Substrat"= oder der Flüssigkeitsoberfläche. Man kann so mit Hilfe des 2DES z.~B. die Änderung der Oberflächeneigenschaften der Heliumoberfläche~\cite{Kon00} oder aber auch die Eigenschaften der Substratoberfläche unter dem Heliumfilm, wie in Kapitel~\ref{sec:theo_zweikomponenten} dieser Arbeit, charakterisieren.
\end{itemize}

In der vorliegenden Arbeit wurde eine neue Messmethode für das 2DES im Hohlraumresonator entwickelt und etabliert, mit deren Hilfe der Zustand des Systems genauer als bisher untersucht werden kann. Zu Beginn wurden Messungen des 2DES auf Bulk"=Helium durchgeführt, welche sich allgemein durch gut reproduzierbares Verhalten auszeichnen. Im Weiteren wurde auf Elektronensysteme auf dünnen Heliumfilmen übergegangen. Da hier die Präparierung der Substrate für den Heliumfilm eine entscheidende Rolle spielt wurden zuerst einfach herzustellende PMMA"=beschichtete Silizium"=Substrate und später \SiO"=bedeckte Substrate für die Messungen verwendet. Letztere erlauben es, sehr hohe Elektronendichten unter reversiblen Bedingungen zu erreichen. Um die Reproduktion der vorgestellten Verfahren und Vorgehensweisen zu erleichtern wurde eine ausführliche Beschreibung der Vorgehensweisen und Abläufe Wert gelegt. Dies soll sicherstellen, dass die vorgestellten Experimente unter vergleichbaren Bedingungen reproduzierbar sind.

Die Suche nach weiteren Hinweisen auf den Phasenübergang des 2DES in ein entartetes Fermi"=Gas, wie sie schon in \cite{guenzler} und \cite{Gue96} vorgestellt wurden, stellt einen interessanten Aspekt der Arbeit dar. Veröffentlichungen wie \cite{Vos98} zeigen auch das allgemeine Interesse an diesem Thema. Generell ist es so, dass sich die Elektronen eines 2DES im Halbleitersystem normalerweise im  Zustand des entarteten Fermi"=Gases befinden und erst eine geschickte \glqq{}Verdünnung\grqq{} des Systems den Phasenübergang erreichbar macht. Im Falle des 2DES auf flüssigem Helium ist die Vorgehensweise genau entgegengesetzt. Da man diese 2DES von einer verschwindenden Elektronendichte ausgehend erzeugen kann, gilt es hier, besonders hohe Dichten zu erreichen.

Teile der Arbeit sind bereits in folgenden Artikeln veröffentlicht:

\selectlanguage{\english}
\begin{description}
	\item[\cite{Ara99}] \textsc{Arai, T.}, \textsc{A.~W{\"u}rl}, \textsc{P.~Leiderer}, \textsc{T.~Shiino}, and\ \textsc{K.~Kono}: {\em {C}hemical reaction between hydrogen atoms and electrons on the surface of superfluid {$^4$}{H}e}.\newblock Physica B, \textbf{284--288}, 164, (1999).
0
	\item[\cite{Shi01}] \textsc{Shikin, V.}, \textsc{J.~Klier}, \textsc{I.~Doicescu}, \textsc{A.~W{\"u}rl}, and\ \textsc{P.~Leiderer}: {\em {D}ip problem of the electron mobility on a thin helium film}. \newblock Phys. Rev. B, \textbf{64}, 73401, (2001).
	
	\item[\cite{Wue01}] \textsc{W{\"u}rl, A.}, \textsc{J.~Klier}, \textsc{P.~Leiderer}, and\ \textsc{V.~Shikin}: {\em {E}lectrons on {L}iquid {H}elium in a {R}esonator}. \newblock J. Low Temp. Phys., \textbf{}, (2001).
	
	\item[\cite{Kli01}] \textsc{Klier, J.}, \textsc{T.~G{\"u}nzler}, \textsc{A.~W{\"u}rl}, \textsc{P.~Leiderer}, \textsc{G.~Mistura}, \textsc{E.~Teske}, \textsc{P.~Wyder}, and\ \textsc{V.~Shikin}: {\em {T}wo-{F}raction {E}lectron {S}ystem on a {T}hin {H}elium {F}ilm}. \newblock J. Low Temp. Phys., \textbf{122}, 451, (2001).
	
	\item[\cite{Kli02}] \textsc{Klier, J.}, \textsc{A.~W{\"u}rl}, \textsc{P.~Leiderer}, \textsc{G.~Mistura}, and\ \textsc{V.~Shikin}: {\em {C}yclotron {R}esonance for 2{D} {E}lectrons on {T}hin {H}elium {F}ilms}. \newblock Phys. Rev. B, \textbf{65}, 165428, (2002).
	
	\item[\cite{Wue03}] \textsc{W{\"u}rl, A.}, \textsc{J.~Klier}, \textsc{P.~Leiderer}, and\ \textsc{V.~Shikin}: {\em {C}yclotron {R}esonance for 2{D} electrons on helium films above rough substrates}. \newblock Physica E, \textbf{18}, 184, (2003).
	
	\item[\cite{Kli04}] \textsc{Klier, J.}, \textsc{A.~W{\"u}rl}, and\ \textsc{V.~Shikin}: {\em {A}nticrossing {P}henomena in a {R}esonator with 2{D} {E}lectrons on {L}iquid {H}elium}. \newblock JETP Letters, \textbf{79}, 218, (2004).
\end{description}
\selectlanguage{\ngerman}

\section*{Allgemeine Hinweise}
Da die Erfassung der Messdaten und aller Parameter dieser Arbeit auf rein elektronischem Wege erfolgte, hier noch einige Hinweise, die der Transparenz der vorgestellten Ergebnisse und deren nachträglicher Überprüfbarkeit dienen. 
Abbildungen dieser Arbeit, denen vom Autor gemessene Daten zu Grunde liegen, tragen Vermerke, welcher Abschnitt aus welchen Datensatz dargestellt ist (Nomenklatur <{\itshape Jahr}>/<{\itshape Monat}> \#<{\itshape lfd.\ Nummer}>) und falls zutreffend sind Hinweise auf eine Nachbearbeitung der Originaldaten vermerkt. Die PDF-Version der Arbeit enthält an diesen Stellen noch Verweise auf eine HTML-Seite die einen Überblick über die Daten der jeweiligen Messung und auch die Originaldaten und "=vermerke enthält. Die im Rahmen dieser Arbeit gewonnenen Messdaten können unter \link{www.wuerl.net/diss} jederzeit abgerufen werden.
